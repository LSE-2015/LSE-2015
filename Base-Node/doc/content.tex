\section{Descripción}
El presente documento tiene como objeto exponer los requisitos, el funcionamiento  y cómo se usa el software del nodo central del juego \textit{Carrera con balizas contra-reloj}.

\section{Requisitos}
El nodo central necesita:
\begin{itemize}

   \item Una Raspberry Pi con la librería de C WiringPi para Rapberry Pi instalada, \textit{www.wiringpi.com}.
   \item Un pulsador tipo \textit{push button}, una resistencia y una \textit{protoboard}.
   \item Cables tipo \textit{jumpers} macho-macho y hembra-macho para realizar las conexiones.

\end{itemize}

El montaje hace referencia al \textit{pinout} de una Raspberry Pi B+. Para otros modelos de Raspberry el \textit{pinout} puede cambiar por lo que las siguientes indicaciones respecto a los interconexión de los pines podrían no ser válidas e incluso será necesario modificar el software.

En la \textit{protoboard} protoboard se conectará el pulsador. Uno de sus extremos se conectará al pin 1 (+3.3V) y el otro se conectará por un lado a una resistencia conectada por el otro extremo al pin 6 (GND) y por otro lado, el otro extremo del pulsador se conectará al pin 7 (GPIO 7).

Los pines 11 (GPIO 0) y 13 (GPIO 2) se conectarán a las señales de interrupción de las balizas. También será necesario conectar las masas de todos los dispositivos, es decir el pin 6 de la Raspberry se conectará también con las masas del las balizas.

Por otra parte será necesario descargar el código referente al nodo central que se encuentra en el repositorio dentro del directorio \textit{Base-Node/src} a la Raspberry. Una vez descargado se ejecuta el comando \textit{make} para compilar el código. Una vez montado \textit{setup} ya se puede ejecutar el software a través del comando \textit{sudo ./nCentral}

\section{Especificaciones}
El juego comienza con un mensaje de bienvenida:
    \begin{description}
    		\item \hspace{10mm} ¡Bienvenido a la contra-reloj de drones!  
    		\item \hspace{10mm} Pulse el botón de 'start' para comenzar.
	\end{description}

Tras este mensaje, el usuario ya puede pulsar el botón de \textit{start} para comenzar el juego. Al pulsar el botón comienza a contar el tiempo, por lo que el jugador deberá colocar el dron previamente en la posición de salida y estar preparado para hacerlo depegar. Se mostrará por pantalla el siguiente mensaje:

    \begin{description}
    		\item \hspace{10mm} El juego ha comenzado. El dron puede despegar.
	\end{description}

Cuando el dron pase por la primera baliza se mostrará el siguiente mensaje por pantalla con el tiempo transcurrido:
    \begin{description}
    		\item \hspace{10mm} El tiempo del 1er 'checkpoint' ha sido: XXX milisegundos
	\end{description}
	
Finalmente cuando el dron pase por la segunda y última baliza el juego terminará y se mostrará el tiempo total que ha tardado el dron en realizar el circuito. A continuación el usuario podrá comenzar una nueva carrera volviendo a pulsar el botón de \textit{start}. El mensaje mostrado por pantalla será:

    \begin{description}
    		\item \hspace{10mm} Juego terminado. El tiempo total ha sido: XXX milisegundos
    		\item \hspace{10mm} Coloque el dron en la posición de salida y pulse el botón de 'start' para que de comienzo una nueva carrera.
	\end{description}	
