\section{Description}
The \textbf{servoctrl} application provides the functionality of controlling up
to two servo motors from a Raspberry Pi embedded system.

\section{Application}
The application receives input on network socket \textbf{2301}. The input
commands have a fixed size of one byte in the following format:

   \subsubsection{\texttt{Input}}
%   \begin{itemize}
%      \item 8 bit.
%      \item \textbf{Write}.
%   \end{itemize}
   \begin{table}[!htb]
      \begin{center}
         \begin{tabular}{|c|c|c|}
            \rowcolor{black}
            \textcolor{white}{\textbf{Bit}}  &
            \textcolor{white}{\textbf{Name}} &
            \textcolor{white}{\textbf{Description}}\\
            \hline
            \hline
            \textbf{6-0} & \texttt{pos} & Offset source \\
            \hline
            \textbf{7}   & \texttt{sel} & Servo selector \\
            \hline
         \end{tabular}
         \caption{\texttt{Input}}
         \label{table:input}
      \end{center}
   \end{table}

The \textit{sel} field determines which servo motor is being addressed and the
\textit{pos} field contains the position to be set; where bX0000000 is the
minimum angular position of the servo and bX1111111 is the maximum angular
position of the servo.\\

This application makes use of the following GPIOs:

   \subsubsection{\texttt{Output}}
%   \begin{itemize}
%      \item 2 GPIOs.
%      \item \textbf{output}.
%   \end{itemize}
   \begin{table}[!htb]
      \begin{center}
         \begin{tabular}{|c|c|c|}
            \rowcolor{black}
            \textcolor{white}{\textbf{GPIO}}  &
            \textcolor{white}{\textbf{Name}} &
            \textcolor{white}{\textbf{Description}}\\
            \hline
            \hline
            \textbf{XX} & \texttt{servo0} & Servo 0 control signal \\
            \hline
            \textbf{XX} & \texttt{servo1} & Servo 1 control signal \\
            \hline
         \end{tabular}
         \caption{\texttt{Output}}
         \label{table:output}
      \end{center}
   \end{table}


\section{Setup}
A set up script is provided at \texttt{../../scripts/setup/servoctrl\_setup.sh}
to ease the building and installation.

If the setup script finished successfully; the binary should be at\\
\texttt{/usr/local/bin/servoctrl} and a UNIX System V init script should be at\\
\texttt{/etc/init.d/servoctrl}.

Aside form the setup script; the source code and makefiles are also provided.

\section{Service}
To start \textbf{servoctrl} as a daemon; execute \texttt{/etc/init.d/servoctrl}
as root. This daemon can take as argument \texttt{start}, \texttt{stop},
\texttt{restart} or \texttt{force-reload}.

\section{Testing}
A test script is provided at \texttt{../../scripts/test/servoctrl\_test.sh} to
test the application. The test script execute a series of test cases and then
asks for user input to determine if the test executed correctly.

If the test was successful, the scripts returns 0, if the test failed a value
different from 0 is returned.

The test sequence currently implemented is the following:

   \begin{enumerate}
      \item servo0 and servo1 are both set to their minimum position.
      \item servo1 progresses from its minimum position to its maximum position.
      \item servo0 progresses from its minimum position to its maximum position.
      \item servo1 progresses from its maximum position to its minimum position.
      \item servo0 progresses from its maximum position to its minimum position.
      \item servo0 and servo1 both progresses together from their minimum
         position to their maximum position.
      \item servo0 progresses from its maximum position to its minimum position.
      \item servo0 progrsses from its minimum position to its maximum position
         while servo1 progresses from its minimum position to its maximum
         position.
   \end{enumerate}

As an example, in the specific case where the servos control the pan/tilt of an
instrument the test should make the instrument follow a square route and then
inscribe a cross within that square route.
