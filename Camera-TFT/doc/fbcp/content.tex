\section{Description}
The \textbf{fbcp} application replicated the HDMI output of the Raspberry Pi on
the TFT LCD touchscreen.

\section{Application}
The application is developed and maintained by Git user tasanakorn. The source
code of the application can be found at:\\

\texttt{https://github.com/tasanakorn/rpi-fbcp.git}\\

The \textbf{fbcp} application copies the content of \texttt{/dev/fb0} (HDMI) to
\texttt{/dev/fb1} (TFT LCD touchscreen). The application is started without
arguments or parameters.

\section{Setup}
A set up script is provided at \texttt{../../scripts/setup/fbcp\_setup.sh} to
ease the building and installation.

If the setup script finished successfully; the binary should be at\\
\texttt{/usr/local/bin/fbcp} and a UNIX System V init script should be at\\
\texttt{/etc/init.d/fbcp}.

\section{Service}
To start \textbf{fbcp} as a daemon; execute \texttt{/etc/init.d/fbcp}
as root. This daemon can take as argument \texttt{start}, \texttt{stop},
\texttt{restart} or \texttt{force-reload}.

\section{Testing}
A test script is provided at
\texttt{../../scripts/test/camera-fbcp-tft\_test.sh} to test the application.
The test script execute a series of test cases and then asks for user input to
determine if the test executed correctly.

If the test was successful, the scripts returns 0, if the test failed a value
different from 0 is returned.

The test sequence currently implemented starts and stops the service while an
instance of \texttt{raspivid} is running.

Please note that for \texttt{raspivid} to run the camera must be present and
some system configuration must have taken place at some time prior to the test.
Please refer to the kernel configuration documentation
(\texttt{../kernel/report.pdf}) for more information on this point.
