\section{Description}
The \textbf{camctrl} application provides the functionality controlling the
pan/tilt mechanism of the camera.
It receives raw commands over UART, process the data and sends processed
commands to the servomotors.

\section{Dependencies}
\textbf{camctrl} depends on \textbf{servoctrl}, which is also provided within
this repository and the documentation of which can be found in
\texttt{../servoctrl}.

\section{Application}
The application receives input through an UART-USB converter. this devices is
assumed to be mapped to \texttt{/dev/ttyUSB0}.

\textbf{camctrl} expects received data to be in the following format:
\textbf{8N1 9600bps}.

The application interprets incoming data as shown in the table below:

   \subsubsection{\texttt{Input}}
%   \begin{itemize}
%      \item 8 bit.
%      \item \textbf{Write}.
%   \end{itemize}
   \begin{table}[!htb]
      \begin{center}
         \begin{tabular}{|c|c|c|}
            \rowcolor{black}
            \textcolor{white}{\textbf{Bit}}  &
            \textcolor{white}{\textbf{Name}} &
            \textcolor{white}{\textbf{Description}}\\
            \hline
            \hline
            \textbf{0}   & \texttt{down}  & State of "down" button \\
            \hline
            \textbf{1}   & \texttt{up}    & State of "up" button \\
            \hline
            \textbf{2}   & \texttt{right} & State of "right" button \\
            \hline
            \textbf{3}   & \texttt{left}  & State of "left" button \\
            \hline
            \textbf{7-4} & \texttt{x} & RESERVED \\
            \hline
         \end{tabular}
         \caption{\texttt{Input}}
         \label{table:input}
      \end{center}
   \end{table}

   \subsubsection{\texttt{Output}}
%   \begin{itemize}
%      \item 2 GPIOs.
%      \item \textbf{output}.
%   \end{itemize}
   \begin{table}[!htb]
      \begin{center}
         \begin{tabular}{|c|c|c|c|c|}
            \rowcolor{black}
            \textcolor{white}{\textbf{BCM GPIO}}  &
            \textcolor{white}{\textbf{Header Pin}} &
            \textcolor{white}{\textbf{Function}} &
            \textcolor{white}{\textbf{Description}}\\
            \hline
            \hline
            \textbf{21} & \textbf{40} &
               \texttt{interupt} & interrupts signal \\
            \hline
         \end{tabular}
         \caption{\texttt{Output}}
         \label{table:output}
      \end{center}
   \end{table}

Please not that this application considers the state bits in the byte to be
active-low; meaning that the least significant nibble of the byte is b1111 when
all of the buttons are at rest.

If the "up", "down", "left" and "right buttons are asserted simultaneously, the
application outputs 100ms pulse on the output interrupt pin that can be used by
external devices to trigger an event.

If, on the other hand, only the "up" and "down" or the "left" and "right"
buttons are asserted simultaneously, a frame of the video stream is saved to
\texttt{/home/pi/capture.jpg}.

All other messages are processed and commands are sent to the \textbf{servoctrl}
over socket \textbf{2301} in the format specified by the documentation of the
aforementioned application.

\section{Setup}
A set up script is provided at \texttt{../../scripts/setup/camctrl\_setup.sh}
to ease the building and installation.

If the setup script finished successfully; the binary should be at\\
\texttt{/usr/local/bin/camctrl.py} and a UNIX System V init script should be
at\\ \texttt{/etc/init.d/camctrl}.

Aside form the setup script; the source code is also provided.

\section{Service} \label{sec:service}
To start \textbf{camctrl} as a daemon; execute \texttt{/etc/init.d/camctrl}
as root. This daemon can take as argument \textbf{start}, \textbf{stop},
\textbf{restart} or \textbf{force-reload}.

\section{Testing}
A test script is provided at \texttt{../../scripts/test/camctrl\_test.sh} to
test the application. The test script execute a series of test cases and then
asks for user input to determine if the test executed correctly.

If the test was successful, the scripts returns 0, if the test failed a value
different from 0 is returned.

To aid with the testing of this application, a helper script has been developed
(\texttt{camctrl-dummy.py}) to simulate UART messages with the same UART-USB
converter. For this to work correctly it is necessary to connect the TX and RX
pins of the converter during the test.

The test sequence currently implemented is the following:

   \begin{enumerate}
      \item At the start, the user is asked if the video stream is being
         displayed correctly.
      \item A simulated "pause" condition is sent.
      \item The user is asked if the video stream has paused.
      \item Simulated messages move the pan/tilt mechanism in a square path.
      \item The user is asked if the pan/tilt mechanism followed expected path.
   \end{enumerate}

\section{Issues}
The following issues have been observed in this module:
   \begin{enumerate}
      \item There is a bug in the sevice script \texttt{/etc/init.d/camctrl} or
         the source code that prevents the application from being daemonized. as
         a workaround, the following command can be executed the start
         \textbf{camctrl} detached:\\
         \texttt{screen -d -m -S camctrl python /usr/local/bin/camctrl.py}
      \item At the time of writing, a recent update of the Raspberry Pi
         bootloader and the third party kernel and packages by Adafruit needed
         to control the LCD-TFT touchcreen have caused the screen to not
         function. This is expected to be resolved by the upstream developer.
   \end{enumerate}

